\documentclass[10pt,a4paper]{article}
\usepackage[utf8]{inputenc}
\usepackage{amsmath}
\usepackage{amsfonts}
\usepackage{amssymb}
\usepackage{hyperref}
\usepackage[T1]{fontenc}
\renewcommand\labelitemii{$\circ$}
\newcommand\tab[1][0.5cm]{\hspace*{#1}}
\title{UNRESP: User Guide to Online Map Creation Tool}
\begin{document}

\maketitle
\tableofcontents

\section{Introduction}
This document describes the process behind the creation of online concentration maps for the UNRESP project.

\section{Files}
The input files required to generate a map are:
\begin{itemize}
\item \textbf{xy\_masaya.dat} -- An ASCII file with two space-separated columns of data (no header) containing the projected x and y coordinates (UTM, km, zone P16) of the computational grid used by CALPUFF.
\item \textbf{concrec******.dat} -- A single-column ASCII file (no header) containing the SO$_2$ concentrations (ug/cm$^3$) output by CALPUFF at each grid point at a particular output time (in the same order as the x,y data file).
\end{itemize}

\section{Python script}
The masaya\_conc.py script should be run from the (UNIX) command line as follows:\\\\
\tab \texttt{\$ python <py-path>/masaya\_conc.py <concFile>}\\\\
where:
\begin{itemize}
\item \texttt{<py-path>} is the path (either relative to the current directory, or full) to the directory in which the python script is located (replace with “./” if it is in the current directory)
\item \texttt{<concFile>} is the path (relative or full) to the concrec data file.
\end{itemize}
The script also assumes that the xy\_masaya.dat data file is in a sub-directory named 'Data' located directly above the directory containing the python script itself (in line with the directory structure of the git repository).\\\\
The script requires a certain number python packages to be imported, including gmplot and utm. If any of these packages are missing on the host computer, the script will currently fail. I (JON) have logged a request with IT to add all required packages to the loadable python3 module (University of Leeds), so that running 'module load python3' from the UNIX command line should allow successful execution of the script. If however the user still experiences import errors, they might want to try installing the missing packages themselves.\\\\
Running the script successfully will generate (in the current directory):
\begin{itemize}
\item A static image file of the S0$_2$ plume (with no basemap) with the naming convention 'map\_concrec******.png'.
\item An interactive webpage of the S0$_2$ plume (with a google-maps-type basemap) with the naming convention 'map\_concrec******.html'.
\end{itemize}
The python script works in the following way:
\begin{itemize}
\item The spatial data file (xy\_masaya.dat) is read in, and converted to lat/lon using the utm package.
\item The concentration data file(s) are then read in and stored into appropriately sized arrays
\item Any concentrations under 20 ug/m$^3$ are ‘masked’ so that they will appear transparent in the plot
\item The concentration data are then ‘binned’ using the following limits:
\begin{itemize}
\item C < 20 (ug/m$^3$)
\item 20 < C < 350
\item 350 < C < 600
\item 600 < C < 2600
\item 2600 < C < 9000
\item 9000 < C < 14000
\item C > 14000
\end{itemize}
Each bin is assigned a different colour from the discrete colour bar, which replicates the limits as shown on this webpage: \url{http://homepages.see.leeds.ac.uk/~earunres/masayaSO2.html}.
\item The pcolormesh tool from matplotlib is used to create the static png image.
\item The gmplot.GoogleMapPlotter function is used to plot each concentration data point (plotted as cell-centred non-overlapping squares) onto the background map using the appropriate lat/lon values and colour.
\end{itemize}



\end{document}